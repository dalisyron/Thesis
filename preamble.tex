\usepackage{amsthm,amssymb,amsmath}
\usepackage{float}
\usepackage[top=40mm, bottom=40mm, left=25mm, right=35mm]{geometry}
\usepackage{graphicx}
\usepackage{framed} 
\usepackage{lastpage}
\usepackage[dvipsnames]{xcolor}
\usepackage{listings}
\usepackage{fontspec}
\usepackage{mdframed}
\usepackage{tikz}
\usepackage{dot2texi}
\usepackage{environ}
\usepackage{multirow}
\usepackage{rotating}

\setcounter{tocdepth}{1}
\newcommand{\pderiv}[2]{\frac{\partial #1}{\partial #2}}
\usetikzlibrary{shapes ,arrows ,positioning}
\makeatletter
\newsavebox{\measure@tikzpicture}
\NewEnviron{scaletikzpicturetowidth}[1]{%
	\def\tikz@width{#1}%
	\def\tikzscale{1}\begin{lrbox}{\measure@tikzpicture}%
		\BODY
	\end{lrbox}%
	\pgfmathparse{#1/\wd\measure@tikzpicture}%
	\edef\tikzscale{\pgfmathresult}%
	\BODY
}
\makeatother

\newmdtheoremenv{defi}{تعریف}[chapter]
\newmdtheoremenv{thm}{قضیه}[chapter]

\setmonofont{JetBrains Mono}[
Contextuals = Alternate,
Ligatures = TeX,
]

% \usepackage[pagebackref=false,colorlinks=false,linkcolor=black,citecolor=blue]{hyperref}
\usepackage[pagebackref=false]{hyperref}
\hypersetup{
	colorlinks   = true, %Colours links instead of ugly boxes
	urlcolor     = black, %Colour for external hyperlinks
	linkcolor    = black, %Colour of internal links
	citecolor   = black %Colour of citations
}

\usepackage{fancyhdr}
\usepackage{setspace}
\usepackage{subfigure}
\usepackage{graphicx}
\usepackage{caption}
\usepackage[subfigure]{tocloft}
\usepackage[nottoc]{tocbibind}
\usepackage{makeidx}
\usepackage{amsmath}
\usepackage{booktabs}
\usepackage{algorithm}
\usepackage{algpseudocode}
\algnewcommand\algorithmicforeach{\textbf{for each}}
\algdef{S}[FOR]{ForEach}[1]{\algorithmicforeach\ #1\ \algorithmicdo}
\makeindex
\renewcommand{\thealgorithm}{\arabic{chapter}.\arabic{algorithm}} 

\usepackage{cases}% http://ctan.org/pkg/cases
\renewcommand{\theequation}{\thesection.\arabic{equation}}
%%%%%%%%%%%%%%%%%%%%%%%%%%
% فراخوانی بسته زی‌پرشین و تعریف قلم فارسی و انگلیسی
\usepackage{xepersian}
\setmainfont[Ligatures=TeX, Path=fonts/, Scale=1.2]{XB Niloofar.ttf}
\settextfont[Scale=1.2, Path=fonts/, BoldFont=B Nazanin Bold.ttf]{B Nazanin.ttf}
\setlatintextfont[Scale=0.8, Path=fonts/, BoldFont=timesbd.ttf]{times.ttf}

%%%%%%%%%%%%%%%%%%%%%%%%%%
% چنانچه می‌خواهید اعداد در فرمول‌ها، انگلیسی باشد، خط زیر را غیرفعال کنید

\setdigitfont[Scale=0.8, Path=fonts/]{XB Zar.ttf}%{Persian Modern}
%%%%%%%%%%%%%%%%%%%%%%%%%%
% تعریف قلم‌های فارسی و انگلیسی اضافی برای استفاده در بعضی از قسمت‌های متن
\defpersianfont\titlefont[Scale=0.8, Path=fonts/]{XB Titre.ttf}
% \defpersianfont\iranic[Scale=1.1]{XB Zar Oblique}%Italic}%
% \defpersianfont\nastaliq[Scale=1.2]{IranNastaliq}

%%%%%%%%%%%%%%%%%%%%%%%%%%
% دستوری برای حذف کلمه «چکیده»
\renewcommand{\abstractname}{}
% دستوری برای حذف کلمه «abstract»
%\renewcommand{\latinabstract}{}
% دستوری برای تغییر نام کلمه «اثبات» به «برهان»
\renewcommand\proofname{\textbf{برهان}}
% دستوری برای تغییر نام کلمه «کتاب‌نامه» به «مراجع»
\renewcommand{\bibname}{مراجع}
% دستوری برای تعریف واژه‌نامه انگلیسی به فارسی
\newcommand\persiangloss[2]{#1\dotfill\lr{#2}\\}
% دستوری برای تعریف واژه‌نامه فارسی به انگلیسی 
\newcommand\englishgloss[2]{#2\dotfill\lr{#1}\\}
% تعریف دستور جدید «\پ» برای خلاصه‌نویسی جهت نوشتن عبارت «پروژه/پایان‌نامه/رساله»
\newcommand{\پ}{پروژه/پایان‌نامه/رساله }
\lstset{
basicstyle = \ttfamily \linespread{1.0} \scriptsize,
columns = flexible,
}
\lstdefinelanguage{Kotlin}{
frame=single,
tabsize=4,
comment=[l]{//},
commentstyle={\color{gray}\ttfamily},
emph={filter, first, firstOrNull, forEach, lazy, map, mapNotNull, println},
emphstyle={\color{OrangeRed}},
identifierstyle=\color{black},
keywords={!in, !is, abstract, actual, annotation, as, as?, break, by, catch, class, companion, const, constructor, continue, crossinline, data, delegate, do, dynamic, else, enum, expect, external, false, field, file, final, finally, for, fun, get, if, import, in, infix, init, inline, inner, interface, internal, is, lateinit, noinline, null, object, open, operator, out, override, package, param, private, property, protected, public, receiveris, reified, return, return@, sealed, set, setparam, super, suspend, tailrec, this, throw, true, try, typealias, typeof, val, var, vararg, when, where, while},
keywordstyle={\color{NavyBlue}\bfseries},
morecomment=[s]{/*}{*/},
morestring=[b]",
morestring=[s]{"""*}{*"""},
ndkeywords={@Deprecated, @JvmField, @JvmName, @JvmOverloads, @JvmStatic, @JvmSynthetic, Array, Byte, Double, Float, Int, Integer, Iterable, Long, Runnable, Short, String, Any, Unit, Nothing},
ndkeywordstyle={\color{BurntOrange}\bfseries},
sensitive=true,
stringstyle={\color{ForestGreen}\ttfamily},
}
\makeatletter
\renewcommand*\verbatim@nolig@list{}
\makeatother
%\newcommand\BackSlash{\char`\\}

%%%%%%%%%%%%%%%%%%%%%%%%%%
\SepMark{-}

% تعریف و نحوه ظاهر شدن عنوان قضیه‌ها، تعریف‌ها، مثال‌ها و ...
\theoremstyle{definition}
\newtheorem{definition}{تعریف}[section]
%\theoremstyle{theorem}
\newtheorem{theorem}[definition]{قضیه}
\newtheorem{lemma}[definition]{لم}
\newtheorem{proposition}[definition]{گزاره}
\newtheorem{corollary}[definition]{نتیجه}
\newtheorem{remark}[definition]{ملاحظه}
\theoremstyle{definition}
\newtheorem{example}[definition]{مثال}

\newcommand{\ImagePage}[1]{
\thispagestyle{empty}
\begin{tikzpicture}[remember picture,overlay]
	\node at (current page.center) {\includegraphics[width=\pdfpagewidth,height=\pdfpageheight]{#1}};
\end{tikzpicture}
\newpage
}
%\renewcommand{\theequation}{\thechapter-\arabic{equation}}
%\def\bibname{مراجع}
\def\listfigurename{فهرست تصاویر}
\def\listtablename{فهرست جداول}

%%%%%%%%%%%%%%%%%%%%%%%%%%%%
% دستورهایی برای سفارشی کردن سربرگ صفحات
% \newcommand{\SetHeader}{
% \csname@twosidetrue\endcsname
% \pagestyle{fancy}
% \fancyhf{} 
% \fancyhead[OL,EL]{\thepage}
% \fancyhead[OR]{\small\rightmark}
% \fancyhead[ER]{\small\leftmark}
% \renewcommand{\chaptermark}[1]{%
	% \markboth{\thechapter-\ #1}{}}
% }
%%%%%%%%%%%%5
%\def\MATtextbaseline{1.5}
%\renewcommand{\baselinestretch}{\MATtextbaseline}
\doublespacing
%%%%%%%%%%%%%%%%%%%%%%%%%%%%%
% دستوراتی برای اضافه کردن کلمه «فصل» در فهرست مطالب

\newlength\mylenprt
\newlength\mylenchp
\newlength\mylenapp

\renewcommand\cftpartpresnum{\partname~}
\renewcommand\cftchappresnum{\chaptername~}
\renewcommand\cftchapaftersnum{:}

\settowidth\mylenprt{\cftpartfont\cftpartpresnum\cftpartaftersnum}
\settowidth\mylenchp{\cftchapfont\cftchappresnum\cftchapaftersnum}
\settowidth\mylenapp{\cftchapfont\appendixname~\cftchapaftersnum}
\addtolength\mylenprt{\cftpartnumwidth}
\addtolength\mylenchp{\cftchapnumwidth}
\addtolength\mylenapp{\cftchapnumwidth}

\setlength\cftpartnumwidth{\mylenprt}
\setlength\cftchapnumwidth{\mylenchp}	

\makeatletter
{\def\thebibliography#1{\chapter*{\refname\@mkboth
		{\uppercase{\refname}}{\uppercase{\refname}}}\list
	{[\arabic{enumi}]}{\settowidth\labelwidth{[#1]}
		\rightmargin\labelwidth
		\advance\rightmargin\labelsep
		\advance\rightmargin\bibindent
		\itemindent -\bibindent
		\listparindent \itemindent
		\parsep \z@
		\usecounter{enumi}}
	\def\newblock{}
	\sloppy
	\sfcode`\.=1000\relax}}
\makeatother

\usepackage{grfext}
\graphicspath{{figures/}}
\usepackage{caption}

\DefaultMathsDigits