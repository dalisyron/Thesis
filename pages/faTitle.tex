% !TeX root=main.tex
% در این فایل، عنوان پایان‌نامه، مشخصات خود، متن تقدیمی‌، ستایش، سپاس‌گزاری و چکیده پایان‌نامه را به فارسی، وارد کنید.
% توجه داشته باشید که جدول حاوی مشخصات پروژه/پایان‌نامه/رساله و همچنین، مشخصات داخل آن، به طور خودکار، درج می‌شود.
%%%%%%%%%%%%%%%%%%%%%%%%%%%%%%%%%%%%
% دانشگاه خود را وارد کنید
\university{علم و صنعت ایران}
% دانشکده، آموزشکده و یا پژوهشکده  خود را وارد کنید
\faculty{دانشکده مهندسی کامپیوتر}
% گروه آموزشی خود را وارد کنید
\department{گروه نرم افزار}
% گروه آموزشی خود را وارد کنید
\subject{مهندسی کامپیوتر}
% گرایش خود را وارد کنید
% عنوان پایان‌نامه را وارد کنید
\title{استراتژی تخلیه محاسباتی احتمالی برای وظایف ناهمگون}
% نام استاد(ان) راهنما را وارد کنید
\firstsupervisor{رضا انتظاری ملکی}
%\secondsupervisor{استاد راهنمای دوم}
% نام استاد(دان) مشاور را وارد کنید. چنانچه استاد مشاور ندارید، دستور پایین را غیرفعال کنید.
%\firstadvisor{استاد مشاور اول}
%\secondadvisor{استاد مشاور دوم}
% نام دانشجو را وارد کنید
\name{محمدمبین}
% نام خانوادگی دانشجو را وارد کنید
\surname{داریوش همدانی}
% شماره دانشجویی دانشجو را وارد کنید
\studentID{96521191}
% تاریخ پایان‌نامه را وارد کنید
\thesisdate{خرداد ۱۴۰۱}
% به صورت پیش‌فرض برای پایان‌نامه‌های کارشناسی تا دکترا به ترتیب از عبارات «پروژه»، «پایان‌نامه» و »رساله» استفاده می‌شود؛ اگر  نمی‌پسندید هر عنوانی را که مایلید در دستور زیر قرار داده و آنرا از حالت توضیح خارج کنید.
%\projectLabel{پایان‌نامه}

% به صورت پیش‌فرض برای عناوین مقاطع تحصیلی کارشناسی تا دکترا به ترتیب از عبارات «کارشناسی»، «کارشناسی ارشد» و »دکترا» استفاده می‌شود؛ اگر  نمی‌پسندید هر عنوانی را که مایلید در دستور زیر قرار داده و آنرا از حالت توضیح خارج کنید.
\degree{کارشناسی}

\firstPage
\besmPage
\davaranPage

%\vspace{.5cm}
% در این قسمت اسامی اساتید راهنما، مشاور و داور باید به صورت دستی وارد شوند
%\renewcommand{\arraystretch}{1.2}
\begin{center}
\begin{tabular}{| p{8mm} | p{18mm} | p{.17\textwidth} |p{14mm}|p{.2\textwidth}|c|}
\hline
ردیف	& سمت & نام و نام خانوادگی & مرتبه \newline دانشگاهی &	دانشگاه یا مؤسسه & امضــــــــــــا\\
\hline
۱  & استاد راهنما & دکتر \newline رضا انتظاری ملکی 
& استادیار & دانشگاه \newline علم و صنعت ایران &  \\
\hline
۲ & استاد داور \newline داخلی	 & دکتر \newline ....  & ..... & 
دانشگاه  \newline علم ‌و صنعت ایران & \\
\hline

\end{tabular}
\end{center}

\esalatPage
\mojavezPage

 \newpage

% 
% % سپاس‌گزاری
% \begin{acknowledgementpage}
% سپاس خداوندگار حکیم را که با لطف بی‌کران خود، آدمی را زیور عقل آراست.
% 
% 
% در آغاز وظیفه‌  خود  می‌دانم از زحمات بی‌دریغ استاد  راهنمای خود،  جناب آقای دکتر ...، صمیمانه تشکر و  قدردانی کنم  که قطعاً بدون راهنمایی‌های ارزنده‌  ایشان، این مجموعه  به انجام  نمی‌رسید.
% 
% از جناب  آقای  دکتر ...   که زحمت  مطالعه و مشاوره‌  این رساله را تقبل  فرمودند و در آماده سازی  این رساله، به نحو احسن اینجانب را مورد راهنمایی قرار دادند، کمال امتنان را دارم.
% 
% همچنین لازم می‌دانم از پدید آورندگان بسته زی‌پرشین، مخصوصاً جناب آقای  وفا خلیقی، که این پایان‌نامه با استفاده از این بسته، آماده شده است و همه دوستانمان در گروه پارسی‌لاتک کمال قدردانی را داشته باشم.
% 
%  در پایان، بوسه می‌زنم بر دستان خداوندگاران مهر و مهربانی، پدر و مادر عزیزم و بعد از خدا، ستایش می‌کنم وجود مقدس‌شان را و تشکر می‌کنم از خانواده عزیزم به پاس عاطفه سرشار و گرمای امیدبخش وجودشان، که بهترین پشتیبان من بودند.
% % با استفاده از دستور زیر، امضای شما، به طور خودکار، درج می‌شود.
% \signature 
% \end{acknowledgementpage}
%%%%%%%%%%%%%%%%%%%%%%%%%%%%%%%%%%%%
% کلمات کلیدی پایان‌نامه را وارد کنید
\keywords{یافتن اجتماعات، شبکه‌های پیچیده، الگوریتم‌های گراف}
%چکیده پایان‌نامه را وارد کنید، برای ایجاد پاراگراف جدید از \\ استفاده کنید. اگر خط خالی دشته باشید، خطا خواهید گرفت.
\fa-abstract{
از زمان معرفی مفهوم پردازش لبه‌ای چند دسترسی توسط \lr{ETSI} یکی از مهم‌ترین چالش‌های این حوزه طراحی استراتژی‌های تخلیه بهینه بوده است. با رشد روز افزون کاربردهای مبتنی بر پردازش ابری و همچنین معرفی اینترنت اشیا، انواع زیادی از کاربردها با نیازمندی‌های منابع گوناگون در لبه شبکه به وجود آمده است. در این مقاله، ابتدا استراتژی تخلیه محاسباتی جدیدی مبتنی بر زنجیره ‌های مارکوف گسسته-زمان معرفی می‌شود که با کمک آن می‌توان تخلیه و اجرای وظایف را به گونه‌ای زمان‌بندی کرد که میزان تاخیر سامانه با وجود محدودیت توان مصرفی کمینه شود. پس از معرفی و شرح استراتژی تخلیه، چارچوبی عملی در زبان \lr{Kotlin} ارائه می‌شود که می‌توان با استفاده از آن استراتژی بهینه را برای شبکه لبه‌ای مورد نظر محاسبه کرد و با کمک شبیه‌سازی، آن را با سایر استراتژی‌ها مقایسه کرد. مقاله فعلی گسترشی بر پژوهش \lr{Liyu} است.
}

\abstractPage

\newpage\clearpage