\chapter{جمع‌بندی و پیشنهاد‌ها}
در این پژوهش روشی برای بدست آوردن استراتژی تخلیه وظیفه با تاخیر کمینه در شرایط حضور چندین نوع وظیفه در محیط رایانش لبه‌ای معرفی شد. عملکرد این الگوریتم نیز با کمک شبیه‌سازی بررسی شد. در طول انجام این پژوهش ایده‌های مختلفی برای بهبود روش ارائه شده به ذهن ما رسید که برخی از آنها مانند بهینه‌سازی‌های معرفی شده در بخش \ref{sec:optim} پیاده‌سازی شدند. اما برخی از این ایده‌ها به مرحله پیاده‌سازی نرسیدند و پژوهش درباره آنها امکان بهبود روش فعلی را فراهم خواهد کرد. \\

یکی از این موارد کاهش تعداد متغیرهای مسئله برنامه‌ریزی خطی
$\mathcal{P}_2$
(رابطه \ref{eq:p2})
 با استفاده از حذف «تک کنش» ها می‌باشد. پیشتر در بخش \ref{sub:reducevariable} به این موضوع اشاره شد که می‌توان متغیرهایی که متناظر با کنش‌های غیر ممکن هستند را از مسئله بهینه‌سازی 
 $\mathcal{P}_2$
 حذف کرد. با استدلالی مشابه این امکان وجود دارد که متغیرهایی که متناظر با تنها کنش ممکن در یک حالت هستند را از الگوریتم حذف کرد، زیرا مقدار این متغیرها در تعیین استراتژی بهینه نقشی ندارد چون احتمال انتخاب آنها همواره ۱ (قطعی) می‌باشد. با این حال حذف این متغیرها بر خلاف بهینه‌سازی \ref{sub:reducevariable} ساختار زنجیره مارکوف را دگرگون خواهد، بنابراین احتمالا نیازمند تغییر توابع انتقال و/یا تغییر شروط \ref{eq:p2} خواهد شد.
\clearpage