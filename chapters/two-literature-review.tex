\chapter{مروری بر ادبیات و کارهای انجام شده}
پژوهش‌های انجام شده در زمینه تخلیه‌ی پردازش\LTRfootnote{Computation Offloading} را می‌توان بر حسب «ویژگی‌های محیط مسئله» و همینطور «الگوریتم استفاده شده برای حل مسئله» دسته‌بندی کرد. در این فصل ابتدا به معرفی این ویژگی‌ها و الگوریتم‌ها می‌پردازیم و سپس برخی از مقالاتی که ارتباط نزدیکی با \CurrentProject دارند را معرفی می‌کنیم.

\section[ویژگی‌های محیط مسئله]{بررسی مقالات از نظر ویژگی‌های محیط مسئله} 
در جدول \ref{table:mohit} که برگرفته از \cite{wang2019} می‌باشد، برخی از ویژگی‌های محیط مسئله و حالت‌های ممکن برای این ویژگی‌ها مشاهده می‌شود.
\begin{table}[H]
	\centering
	\begin{latin}

\begin{tabular}{@{}lllll@{}}
	\toprule
	\textbf{Granularity} & \textbf{User Count} & \textbf{Mobility} & \textbf{Destination} & \textbf{Metric} \\ \midrule
	Application          & Single UE             & Cloud Server      & Single-server        & Delay           \\
	Task                 & Multi UE   & Edge Server       & Multi-server         & Energy          \\
	Method               &             & Ad hoc            &                      & Cost            \\ \bottomrule
\end{tabular}
	\end{latin}
	\caption{تقسیم‌بندی شرایط محیطی مسئله‌ی تخلیه‌ی پردازش}
	\label{table:mohit}
\end{table}

\subsection*{دانه‌بندی (\lr{Granularity})}
دانه‌بندی به نوع مولفه‌های پردازشی قابل تخلیه در سیستم اشاره دارد. طبق \cite{wang2019} دانه‌بندی را با سه دسته مختلف (به ترتیب از دانه‌ ریز به دانه ‌درشت) بیان می‌کنیم: کاربرد، وظیفه، متد. هر چه دانه‌بندی ریزتر باشد انعطاف‌پذیری سیستم تخلیه بیشتر خواهد بود، به طوری که به توسعه‌دهندگان نرم‌افزار اجازه خواهد داد تا به طور دقیق مشخص کنند که کدام قسمت‌ها از یک کاربرد خاص تخلیه شوند و کدام قسمت‌ها نشوند. با این حال پیاده‌سازی سیستم‌های تخلیه‌ی پردازش به صورت دانه‌ریز به مراتب پیچیده‌تر است. پیاده‌سازی‌های دانه‌ریز همچنین هزینه اضافه\LTRfootnote{Overhead} بیشتری برای ساخت محیط‌های مجازی در سرور خواهند داشت. خلاصه‌ای از انواع دانه‌بندی‌ها در شکل \ref{fig:granularity} آورده شده است.
\begin{figure}[H]
	\centering
	\includegraphics[width=0.4\textwidth]{figures/granularity.png}
	\caption{سه دانه‌بندی مختلف در سامانه‌ی تخلیه‌ی پردازش}
	\label{fig:granularity}
\end{figure}
به عنوان نمونه در \cite{maui} دانه‌بندی در سطح متد صورت گرفته است، در حالی که در \cite{Liu} دانه‌بندی در سطح وظیفه صورت گرفته است. ما نیز در \CurrentProject مسئله‌ی تخلیه‌ی پردازش را در سطح وظیفه حل کرده‌ایم.
\newpage
\subsection*{تعداد کاربران (\lr{User Count})}
در برخی از مقالات مانند \cite{Liu} مسئله‌ی تخلیه‌ی پردازش تنها برای یک کاربر در نظر گرفته می‌شود در حالیکه در برخی از پژوهش‌ها مانند \cite{multiuser} از چندین کاربر همزمان نیز پشتیبانی می‌شود. در \CurrentProject تعداد کاربران را برای سادگی بیشتر یک در نظر می‌گیریم.

\subsection*{تحرک‌پذیری (\lr{Mobility})}
به انجام پردازش توسط هر گره\LTRfootnote{Node}ای به جز گره ایجاد کننده وظیفه، تخلیه‌ی وظیفه گفته می‌شود. طبق این تعریف سه نوع از تحرک‌پذیری را متناظر با نوع گره پردازشی می‌توانیم در نظر بگیریم.
\begin{enumerate}
	\item پردازش در سرور ابری
	\item پردازش در سرور لبه‌ای
	\item پردازش در شبکه‌ای بدون ساختار\LTRfootnote{Ad-hoc} (از دستگاه‌های کاربر)
\end{enumerate}

\subsection*{تعداد سرور}
مشابه با تعداد کاربران، تعداد سرورهای پردازشی در سامانه تخلیه نیز می‌تواند یک یا بیشتر باشد. برای نمونه در \cite{multiuser} مسئله‌ی تخلیه‌ی پردازش برای چندین سرور بررسی شده است. در \CurrentProject ما حالت تک سرور را در نظر می‌گیریم.

\subsection*{معیار بهینه‌سازی}
 معیار بهینه‌سازی به کمیتی اشاره دارد که استراتژی تخلیه‌ی پردازش سعی در بهینه‌سازی آن دارد. برخی از معیارهای رایج عبارتند از: تاخیر، انرژی، کیفیت سرویس، و هزینه. برای مثال در \cite{Liu} معیار تاخیر، در \cite{metricenergy} معیار انرژی و در \cite{metriccost} معیار هزینه در نظر گرفته شده است.
\section{بررسی مقالات از نظر روش حل مسئله}
در جدول \ref{table:algorithms} که برگرفته از 
\cite{Shakarami2020}
می‌باشد، یک دسته‌بندی کلی از الگوریتم‌های رایج در حل مسئله‌ی تخلیه‌ی پردازش مشاهده می‌شود. برای آشنایی بیشتر با این روش ها به
\cite{wang2019}
و \cite{Shakarami2020} رجوع شود. در \CurrentProject ما از \textbf{الگوریتمی }قطعی\LTRfootnote{Deteministic} بر پایه برنامه‌ریزی خطی برای یافتن \textbf{استراتژی تخلیه} تصادفی\LTRfootnote{Stochastic} استفاده می‌کنیم. 

% Please add the following required packages to your document preamble:
% \usepackage{booktabs}
\begin{table}[H]
	\centering
	\begin{latin}
		\begin{tabular}{@{}ll@{}}
			\toprule
			\textbf{Model} & \textbf{Examples}                                                                                                                                                           \\ \midrule
			Stochastic     & \begin{tabular}[c]{@{}l@{}}Machine learning, \\ Generalized poison distribution, \\ Game theory, \\ Queuing theory, \\ Markov processes, \\ Gaussian processes\end{tabular} \\ \midrule
			Deterministic  & \begin{tabular}[c]{@{}l@{}}Some supervised Machine Learning approaches (e.g., KNN),\\ Linear and non-linear programming, \\ Linear regression equation\end{tabular}        
		\end{tabular}
	\end{latin}
	\caption{تقسیم‌بندی الگوریتم‌های حل مسئله‌ی تخلیه‌ی پردازش}
	\label{table:algorithms}
\end{table}

\section{پژوهش‌های مرتبط}
در \cite{Liu} مسئله‌ی تخلیه‌ی وظیفه با تاخیر کمینه با استفاده از روشی مبتنی بر زنجیره‌ی مارکوف و برنامه‌ریزی خطی حل شده است. در \CurrentProject محیط تک کاربر و تک سرور در نظر گرفته شده است. روش ارائه شده در درازمدت عملکرد بهینه دارد اما چندین کاستی دارد از جمله عدم پشتیبانی از وظایف با نیازمندی‌های پردازشی متفاوت و عدم پشتیبانی از موازی‌سازی. \CurrentProject گسترشی بر این مقاله است. \\

در \cite{samanta} یک مکانزیم تخلیه‌ی وظیفه با هزینه کمینه برای محیط رایانش لبه‌ای متحرک ارائه شده است. محیط در نظر گرفته شده از نظر ثابت بودن طول بازه‌های زمانی و تفاوت وظایف و همچنین نحوه‌ی تعریف مسئله‌ی بهینه‌سازی، شبیه به پژوهش ما می‌باشد. اما از نظر معیار و تعداد سرور متفاوت می‌باشد. روش بهینه‌سازی استفاده شده در \CurrentProject روش «ضرایب لاگرانژ» می‌باشد که عملکرد سریعی دارد اما لزوما جواب بهینه سراسری را پیدا نمی‌کند و فقط جواب‌های بهینه محلی را پیدا می‌کند. \\

در \cite{kwak} و \cite{jiang} مشابه با \CurrentProject، ناهمگونی وظایف و تقابل\LTRfootnote{Tradeoff} تاخیر و انرژی در نظر گرفته شده است. با این تفاوت که در این دو مقاله از روش بهینه‌سازی لیاپانوف استفاده شده است. همچنین این دو مقاله مسئله‌ی تخلیه‌ی وظیفه را در محیط رایانش ابری را در نظر گرفته‌اند و نه رایانش لبه‌ای و همچنین چارچوب نرم‌افزاری ای برای حل مسئله در محیط‌های خاص ارائه نداده اند. \\

در \cite{zhang2013} مسئله‌ی تخلیه و زمان‌بندی ارسال و اجرا به صورت همزمان در نظر گرفته شده است و کانال بیسیم به صورت تصادفی مدل شده است و از این ابعاد به پژوهش ما شباهت دارد. معیار بهینه‌سازی در \CurrentProject انرژی مصرفی است. روش ارائه شده عملکرد خوبی دارد و به میزان قابل توجی در انرژی مصرفی صرفه‌جویی می‌کند. با این حال مدل در نظر گرفته شده کاستی‌هایی دارد. یک ایراد اصلی فرض وجود تنها یک کاربرد در سیستم است. به عبارت دیگر تاخیر ایجاد شده به واسطه انتظار کاربردها در صف در نظر گرفته نشده است.

\clearpage
