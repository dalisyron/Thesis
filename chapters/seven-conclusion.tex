\mchapter{جمع‌بندی و پیشنهاد‌ها}
در این پروژه روشی برای بدست آوردن استراتژی تخلیه‌ی وظیفه‌ی تاخیر-کمینه در شرایط حضور چندین نوع وظیفه معرفی شد. همچنین چارچوب نرم‌افزاری جدیدی معرفی شد که قادر به محاسبه‌ی استراتژی تخلیه‌ی وظیفه‌ی بهینه و شبیه‌سازی آن می‌باشد. روش ارائه‌شده به طور جامع تحت آزمایش قرار گرفت و عملکرد آن با کمک شبیه‌سازی بررسی شد. روش ارائه‌شده این پتانسیل را دارد که نحوه‌ی زمان‌بندی وظایف در محیط‌های با وظایف گوناگون مانند اینترنت اشیاء را به طور قابل توجهی بهبود ببخشد. با این حال چندین محدودیت در \CurrentProject وجود دارد، که رفع آنها نیاز به پژوهش بیشتر دارد. در ادامه به برخی از این موارد به طور مختصر اشاره می‌کنیم.

\section{بهبود کارآیی الگوریتم}
یک محدودیت اصلی در روش ارائه‌شده، افت کارآیی الگوریتم با افزایش تعداد صف‌ها می‌باشد. در فصل ۶ اشاره شد که این افت به دلیل انفجار فضای حالت مسئله می‌باشد. به عبارت دیگر با افزایش تعداد متغیرهای مسئله‌ی برنامه‌ریزی خطی
$\mathcal{P}_2$
(رابطه‌ی \ref{eq:p2}) زمان اجرای الگوریتم به صورت تصاعدی بالا می‌رود. در بخش \ref{sub:reducevariable} روشی ارائه شد که تا حدی اندازه‌ی فضای حالت مسئله را کاهش می‌داد. با این حال همانطور که در نتایج شبیه‌سازی مشاهده شد، الگوریتم ارائه‌شده همچنان برای طول صف‌های بیشتر از ۵ کارآیی ندارد. در بخش فعلی دو ایده مختلف را ارائه می‌کنیم که با پژوهش بیشتر درباره آنها شاید بتوان عملکرد الگوریتم را بهبود ببخشید.
\subsection{حذف تک‌کنش‌ها}
یک ایده‌ برای التیام مشکل انفجار فضای حالت، حذف «تک کنش» ها از فضای مسئله ($|S| \times |A|$) می‌باشد. به طور دقیق‌تر کنش $a$ را برای حالت $\tau$ یک \textbf{تک‌کنش} می‌نامیم اگر تنها کنش ممکن در حالت $\tau$ باشد. با توجه به اینکه کنش \texttt{\footnotesize NoOperation} در همه حالت‌ها ممکن است، تک کنش‌ها همواره متناظر با \texttt{\footnotesize NoOperation} می‌باشند. \\

پیشتر در بخش \ref{sub:reducevariable} به این موضوع اشاره شد که می‌توان متغیرهایی که متناظر با کنش‌های غیر ممکن هستند را از مسئله‌ی بهینه‌سازی 
 $\mathcal{P}_2$
 حذف کرد. با استدلالی مشابه این احتمال وجود دارد که بتوان متغیرهایی که متناظر با تنها کنش ممکن در یک حالت هستند را از الگوریتم برنامه‌ریزی خطی حذف کرد، زیرا احتمال انتخاب کنش‌های متناظر با چنین متغیرهایی همواره ۱ (قطعی) می‌باشد و مقدار آن متغیرها در تعیین استراتژی بهینه تاثیری ندارد. \\ 
 
 با دقت در فضای حالت مسئله می‌توان دریافت که بسیاری از حالت‌های فضای مسئله دارای تک‌کنش می‌باشند. برای مثال هر حالتی که پردازنده و واحد ارسال هر دو در آن مشغول باشند از این نوع خواهد بود. فرآوانی چنین حالت‌هایی در فضای حالت مسئله بدین معنی است که حذف آنها می‌تواند کارآیی الگوریتم ارائه‌شده را به طور قابل توجهی بهبود بخشد. با این حال حذف تک‌کنش‌ها از لیست متغیرهای مسئله، بر خلاف بهینه‌سازی \ref{sub:reducevariable} ساختار زنجیره‌ی مارکوف را دگرگون خواهد، بنابراین احتمالا نیازمند تغییر توابع انتقال و تغییر شروط رابطه‌ی \ref{eq:p2} خواهد بود. پژوهش بیشتر در این زمینه مشکل انفجار فضای حالت را به طور کامل حل نخواهد کرد، اما امید می‌رود که تعداد صف‌های قابل پشتیبانی در روش ارائه‌شده را به طور قابل توجهی افزایش دهد.

\subsection{اعمال جریمه برای اتلاف وظیفه}
یک راه بنیادی‌تر برای رفع مشکل انفجار فضای حالت این است که کلا مدل حالت ($\tau$) مسئله را اینگونه تغییر دهیم که هر صف ظرفیتی برابر با \textbf{۲} داشته باشد. طبیعتا اعمال چنین محدودیتی در مسئله فعلی ممکن نیست زیرا فرض کرده‌ایم که اتلاف صورت نمی‌گیرد، به این صورت که حضور در حالت‌های با صف پر ($q_i = Q$) را به منزله غیر کارامد بودن استراتژی در نظر گرفتیم (رجوع شود به بخش \ref{sec:effectiveness}). اما می‌توان تغییری در مدل اعمال کرد که هزینه اتلاف وظیفه را نیز در نظر بگیرد. به طور دقیق‌تر مدل برنامه‌ریزی خطی می‌تواند به ازای یک استراتژی تخلیه داده شده و نرخ ورود $\alpha_1 \cdots \alpha_k$ احتمال وقوع اتلاف وظیفه در صف با طول ۲ را در نظر بگیرد. در چنین حالتی می‌توان به میزان مشخصی مانند $\Omega$ جریمه تاخیر در تابع هدف در نظر گرفت. بدین صورت حل کننده خطی در راستای کم کردن تاخیر، سعی خواهد کرد که احتمال وقوع اتلاف وظیفه را در سامانه پایین بیاورد. با استفاده از این تغییر فضای حالت مسئله به طور قابل توجهی کوچک خواهد شد به طوری که اگر در روش قدیمی $|S_1|$ حالت وجود داشته باشد، در روش جدید 
$\frac{|S_1| \cdot 2^k}{Q^k}$ 
حالت وجود خواهد داشت.
\section{موضع‌گیری استراتژی تخلیه‌ی وظیفه}
یک موضوع مهم که در \CurrentProject به آن پرداخته نشد، نحوه‌ی موضع‌گیری (\lr{Deployment}) استراتژی تخلیه‌ی وظیفه یا به عبارتی «سازوکار تنظیم استراتژی در دستگاه کاربر» است. همانطور که در شبیه‌سازی های انجام شده در فصل ۶ مشاهده شد، الگوریتم محاسبه‌ی استراتژی تخلیه‌ی بهینه به منابع محاسباتی زیادی نیاز دارد. طبیعتا انجام چنین محاسباتی بر روی دستگاه کاربر که موجودیتی مانند تلفن همراه یا اینترنت اشیاء است عملی نخواهد بود. چه بسا که در \CurrentProject نیز برای تمام شبیه‌سازی‌ها از سروری قدرتمند با ۲۴ هسته پردازشی استفاده شد. در سطح بالا، یک راه حل برای موضع‌گیری روش پیشنهادی در پروژه، استفاده از معماری مشابه با شبکه‌های مبتنی بر نرم افزار\LTRfootnote{Software Defined Networks} می‌باشد. برای مثال می‌توان در هر سرور رایانش لبه‌ای فرآیندی را اجرا کرد، که هر $n$ ساعت یک بار، اطلاعات محیطی (مانند نرخ ورود هر نوع از وظایف) را از دستگاه‌های کاربر سرویس‌گیرنده بگیرد و متناسب با شرایط هر محیط، استراتژی تخلیه‌ی بهینه را محاسبه کرده و برای دستگاه کاربر مربوطه ارسال کند. با این وجود، پیاده‌سازی کارآمد چنین سازوکاری، نیازمند پژوهش بیشتر می‌باشد.
\clearpage