\chapter{مقدمه} 
\pagenumbering{arabic}
افزایش روز افزون تعداد دستگاه‌های موجود در لبه شبکه در سال‌های اخیر، و همچنین معرفی کاربردهای نرم افزاری جدید که نیازمند منابع محاسباتی بالا هستند باعث شده است که تقاضای زیادی برای خدمات پردازش ابری بوجود بیاید. پردازش ابری این امکان را به دستگاه‌های هوشمند از جمله تلفن همراه و اینترنت اشیاء می‌دهد که بخشی از پردازش‌های سنگین خود را به سرورهای قدرتمند «تخلیه» کنند تا بر محدودیت‌های پردازشی خود غلبه کنند و کاربردهای نرم افزاری پیچیده‌ای مانند واقعیت افزوده و خودروهای هوشمند را برای کاربران فراهم کنند. \\

با این وجود، پیاده‌سازی‌های سنتی پردازش ابری یک ایراد ذاتی دارند، و آن فاصله زیاد سرورهای ابری با دستگاه‌های پایانی است. الگوی «رایانش لبه‌ای» و معماری‌های استاندارد آن مانند رایانش لبه‌ای دسترسی-چندگانه که توسط سازمان \lr{ETSI} ارائه شده است، سعی دارند تا با آوردن بخشی از منابع محاسباتی به نزدیکی لبه شبکه، این مشکل را تا حدی برطرف کنند. علاوه بر تمایل دستگاه‌های لبه شبکه به کمتر شدن این فاصله و به عبارتی «کشش» منابع محاسباتی توسط آن‌ها به منظور افزایش کیفیت سرویس، شرکت‌های ارائه‌دهنده خدمات ابری نیز تمایل دارند تا با «فشردن» بخشی از منابع محاسباتی خود به لبه شبکه، بار محاسباتی و هزینه‌های تجهیزاتی خود را کاهش دهند. \cite{edgevisions} در جدول \ref{table:compare} مقایسه‌ای کلی از رایانش ابری و لبه‌ای ارائه شده است. با دقت در این جدول می‌توان فهمید که رایانش لبه‌ای جایگزین رایانش ابری نمی‌باشد بلکه مکمل آن است.
\newpage
% Please add the following required packages to your document preamble:
% \usepackage{booktabs}
\begin{table}[H]
	\centering
	\begin{tabular}{@{}lll@{}}
		\toprule
		ویژگی    & رایانش ابری         & رایانش لبه‌ای \\ \midrule
		استقرار          & مرکزی & توزیع شده    \\
		فاصله تا دستگاه کاربر  & زیاد        & کم            \\
		تاخیر             & زیاد        & کم            \\
		تغییرات تاخیر\footnotemark              & زیاد        & کم            \\
		منابع پردازشی & فرآوان       & محدود        \\
		فضای ذخیره‌سازی    & فرآوان       & محدود        \\ \bottomrule
	\end{tabular}
	\caption{مقایسه رایانش ابری و لبه‌ای}
	\label{table:compare}
\end{table}
\LTRfootnotetext[1]{Jitter}
یک امر مهم در پیاده‌سازی کارآمد رایانش لبه‌ای، طراحی استراتژی‌های تخلیه‌ی وظیفه به صورت هوشمند و موثر است. این استراتژی‌ها نحوه تخصیص منابع توسط دستگاه کاربر\LTRfootnote{User Equipment} را مشخص می‌کنند و این امکان را به دستگاه کاربر می‌دهند تا درباره تخلیه یا عدم تخلیه‌ی وظایف محاسباتی در طول زمان تصمیم بگیرد. \\

استراتژی تخلیه بهینه به استراتژی تخلیه‌ای گفته می‌شود که یک تابع «هدف» خاص را تحت شرایط محیطی مشخص کمینه یا بیشینه کند. توابع هدف عموما بر حسب یک یا چند معیار سیستم تعریف می‌شوند. برخی از این معیارها عبارتند از:
\begin{itemize}
	\item تاخیر سرویس (\lr{Service Delay})
	\item توان مصرفی (\lr{(Consumed Power)})
	\item تقدم و تاخر (\lr{Jitter})
	\item هزینه (\lr{Cost})
\end{itemize}

مقدار توابع هدف و شروط مسئله تخلیه بستگی به پارامترهای زیادی دارند، از جمله میزان منابع موجود در دستگاه کاربر، نیازمندی‌های کاربر، کیفیت شبکه دسترسی و شلوغی سرورهای رایانش لبه‌ای. علاوه بر پارامترهای محیطی، ساختار کاربردهای\LTRfootnote{Application} نرم‌افزاری مورد بررسی نیز در مسئله تاثیر می‌گذارند. برای مثال ممکن است که تمام یا بخشی از یک کاربرد خاص قابلیت تخلیه نداشته باشد.  \\

در \CurrentProject \textbf{تاخیر سرویس} به عنوان تابع هدف در نظر گرفته شده است. تاخیر همواره یک معیار اصلی در سنجش کیفیت سامانه‌های کامپیوتری بوده است. همچنین با رشد روز افزون صنعت اینترنت اشیاء، کاربردهای جدیدی در سطح شبکه به وجود آمده است که نیازمندی‌های تاخیر بسیار پایین دارند، به طوری که سرورهای رایانش ابری پاسخگوی این نیازمندی نخواهد بود. و از طرفی نیازمندی‌های پردازشی بالا امکان اجرای این کاربردها به صورت محلی و بهنگام\LTRfootnote{Real-time} را نمی‌دهد. یک نمونه از این دسته از کاربردها «سامانه مدیریت ترافیک هوشمند» است که نیازمند انجام پردازش‌های سنگین در زمان بسیار کم می‌باشد. \\

تاخیر سرویس بسته به اجرای محلی و یا تخلیه از مولفه‌های متفاوتی تشکیل می‌شود. در صورت اجرای محلی تاخیر سرویس از موارد زیر تشکیل خواهد بود:
\begin{enumerate}
	\item تاخیر انتظار در صف وظیفه $d_q$
	\item تاخیر اجرا به صورت محلی $d_{l o c}$
\end{enumerate}
و در صورت تخلیه از موارد زیر تشکیل خواهد شد:
\begin{enumerate}
	\item تاخیر صف $d_q$
	\item تاخیر ارسال $d_{t x}$
	\item تاخیر انتشار $d_{p r o p a g a t i o n}$
	\item تاخیر اجرا در سرور لبه‌ای $d_{s e r v e r}$
	\item تاخیر بازدریافت وظیفه از سرور $d_{r x}$
\end{enumerate}
\newpage
در \CurrentProject روشی برای بدست آوردن استراتژی تخلیه‌ی وظیفه با تاخیر کمینه تحت محدودیت توان مصرفی در محیط رایانش لبه‌ای ارائه خواهیم داد. روش ارائه شده مبتنی بر زنجیره مارکوف گسسته-زمان و برنامه‌ریزی خطی می‌باشد و گسترشی بر روش ارائه شده در \cite{Liu} می‌باشد. نوآوری و مزیت اصلی روش پیشنهادی ما نسبت به مقاله ذکر شده قابلیت پشتیبانی از وظایف با نیازمندی‌های پردازشی و شبکه‌ای متفاوت (وظایف ناهمگون) می‌باشد. انگیزه اصلی از این گسترش، تنوع محاسباتی وظایف در محیط‌های اینترنت اشیاء بوده است. به طور مثال در بسیاری از پژوهش‌های حوزه تخلیه‌ی وظیفه در اینترنت اشیاء، وظایف به دو دسته «سبک» و «سنگین» تقسیم می‌شوند. \cite{yousefpour} \cite{tran} برای درک مفهوم وظایف سبک و سنگین می‌توان مثال اتومبیل خودران را در نظر گرفت. در این کاربرد، وظیفه پردازش اطلاعات تصاویر به منظور راندن خودرو یک وظیفه سنگین محسوب می‌شود، در حالی که وظیفه‌ روشن کردن سیستم گرمایشی خودرو بر حسب داده‌ی سنسور دما، یک وظیفه سبک محسوب می‌شود. \\

ادامه \CurrentProject به پنج فصل تقسیم شده است. در فصل ۲ پژوهش‌های مرتبط انجام شده را مرور می‌کنیم. در فصل ۳ به شرح مسئله تخلیه‌ی وظیفه و ساختار رایانش لبه‌ای می‌پردازیم. در فصل ۴ روش پیشنهادی برای بدست آوردن استراتژی تخلیه بهینه را شرح می‌دهیم. در فصل ۵ ابتدا ساختار نرم‌افزاری\LTRfootnote{Framework} جدیدی مبتنی بر زبان \lr{Kotlin} با نام \lr{Kompute} ارائه می‌دهیم که این امکان را به کاربران و پژوهشگران می‌دهد تا استراتژی تخلیه بهینه را به ازای سیستم دلخواه خود محاسبه کنند و آن استراتژی را با سایر استراتژی‌های پایه\LTRfootnote{Baseline} مقایسه کنند. در بخش دوم از فصل ۵ با استفاده از \lr{Kompute} به آزمایش و شبیه‌سازی روش ارائه شده در بخش ۴ می‌پردازیم. در انتها در فصل ۶ یک جمع‌بندی کلی از تمامی مطالب ارائه می‌دهیم و پیشنهاداتی نیز برای گسترش روش پیشنهادی ارائه می‌کنیم.


\clearpage
