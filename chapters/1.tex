\chapter{مقدمه} 
\pagenumbering{arabic}
افزایش روز افزون تعداد دستگاه‌های موجود در لبه شبکه در سال‌های اخیر، و همچنین معرفی کاربردهای نرم افزاری جدید که نیازمند منابع محاسباتی بالا هستند باعث شده است که تقاضای زیادی برای خدمات پردازش ابری بوجود بیاید. پردازش ابری این امکان را به دستگاه‌های هوشمند از جمله تلفن همراه و اینترنت اشیا می‌دهد که بخشی از پردازش‌های سنگین خود را به سرورهای قدرتمند «تخلیه» کنند تا بر محدودیت‌های پردازشی خود غلبه کنند و کاربردهای نرم افزاری پیچیده‌ای مانند واقعیت افزوده و خودروهای هوشمند را برای کاربران فراهم کنند. \\

با این وجود، پیاده‌سازی‌های سنتی پردازش ابری یک ایراد ذاتی دارند، و آن فاصله زیاد سرورهای ابری با دستگاه‌های پایانی است. معماری رایانش لبه‌ای و پیاده‌سازی‌های استاندارد آن مانند رایانش لبه‌ای دسترسی-چندگانه که توسط سازمان \lr{ETSI} ارائه شده است، سعی دارند تا با آوردن بخشی از منابع محاسباتی به نزدیکی لبه شبکه، این مشکل را تا حدی برطرف کنند. علاوه بر تمایل دستگاه‌های لبه شبکه به کمتر شدن این فاصله و به عبارتی «کشش» منابع محاسباتی توسط آن‌ها به منظور افزایش کیفیت سرویس، شرکت‌های ارائه‌دهنده خدمات ابری نیز تمایل دارند تا با «فشردن» بخشی از منابع محاسباتی خود به لبه شبکه، بار محاسباتی و هزینه‌های تجهیزاتی خود را کاهش دهند. \cite{edgevisions}
\newpage
یک امر مهم در پیاده‌سازی کارآمد رایانش لبه‌ای، طراحی استراتژی‌های تخلیه وظایف هوشمند و موثر است. این استراتژی‌ها نحوه تخصیص منابع توسط دستگاه کاربر\LTRfootnote{User Equipment} را مشخص می‌کنند و این امکان را به دستگاه کاربر می‌دهند تا درباره تخلیه یا عدم تخلیه وظایف محاسباتی در طول زمان تصمیم بگیرد. \\

در این مقاله روشی برای بدست آوردن استراتژی تخلیه بهینه ارائه می‌دهیم که مبتنی بر زنجیره مارکوف گسسته-زمان و برنامه‌ریزی خطی می‌باشد. روش پیشنهادی گسترشی بر روش ارائه شده در \cite{Liu} می‌باشد. نوآوری و مزیت اصلی روش پیشنهادی ما نسبت به مقاله ذکر شده قابلیت پشتیبانی از وظایف با نیازمندی‌های پردازشی و شبکه‌ای متفاوت (وظایف ناهمگون) می‌باشد. انگیزه اصلی از این گسترش، تنوع محاسباتی وظایف در محیط‌های اینترنت اشیا بوده است. به طور مثال در بسیاری از پژوهش‌های حوزه تخلیه وظیفه در اینترنت اشیا، وظایف به دو دسته «سبک» و «سنگین» تقسیم می‌شوند. \cite{yousefpour} \cite{tran} برای درک مفهوم وظایف سبک و سنگین می‌توان مثال اتومبیل خودران را در نظر گرفت. در این کاربرد، وظیفه پردازش اطلاعات تصاویر به منظور راندن خودرو یک وظیفه سنگین محسوب می‌شود، در حالی که وظیفه‌ روشن کردن سیستم گرمایشی خودرو بر حسب داده‌ی سنسور دما، یک وظیفه سبک محسوب می‌شود. \\

ادامه این مقاله به پنج فصل تقسیم شده است. در فصل ۲ به شرح مسئله تخلیه وظیفه و ساختار رایانش لبه‌ای می‌پردازیم. در فصل ۳ پژوهش‌های مرتبط انجام شده را مرور می‌کنیم. در فصل ۴ روش پیشنهادی برای بدست آوردن استراتژی تخلیه بهینه را شرح می‌دهیم. در فصل ۵ ابتدا ساختار نرم‌افزاری\LTRfootnote{Framework} جدیدی مبتنی بر زبان \lr{Kotlin} با نام \lr{Kompute} ارائه می‌دهیم که این امکان را به کاربران و پژوهشگران می‌دهد تا استراتژی تخلیه بهینه را به ازای سیستم دلخواه خود محاسبه کنند و آن استراتژی را با سایر استراتژی‌های پایه\LTRfootnote{Baseline} مقایسه کنند. در بخش دوم از فصل ۴ با استفاده از \lr{Kompute} به طور جامع به آزمایش و شبیه‌سازی روش ارائه شده در بخش ۴ می‌پردازیم. در انتها در فصل ۶ یک جمع‌بندی کلی از تمامی مطالب ارائه می‌دهیم و پیشنهاداتی نیز برای گسترش روش پیشنهادی ارائه می‌کنیم.


\clearpage
