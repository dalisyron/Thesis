\chapter{مقدمه} 
\pagenumbering{arabic}
افزایش روز افزون تعداد دستگاه‌های موجود در لبه شبکه در سال‌های اخیر، و همچنین معرفی کاربردهای نرم افزاری جدید که نیازمند منابع محاسباتی بالا هستند باعث شده است که تقاضای زیادی برای خدمات پردازش ابری بوجود بیاید. پردازش ابری این امکان را به دستگاه‌های هوشمند از جمله تلفن همراه و اینترنت اشیا می‌دهد که بخشی از پردازش‌های سنگین خود را به سرورهای قدرتمند «تخلیه» کنند تا بر محدودیت‌های پردازشی خود غلبه کنند و کاربردهای نرم افزاری پیچیده‌ای مانند واقعیت افزوده و خودروهای هوشمند را برای کاربران فراهم کنند. \\

با این وجود، پیاده‌سازی‌های سنتی پردازش ابری یک ایراد ذاتی دارند، و آن فاصله زیاد سرورهای ابری با دستگاه‌های پایانی است. معماری پردازش لبه‌ای و پیاده‌سازی‌های استاندارد آن مانند پردازش لبه‌ای دسترسی-چندگانه که توسط سازمان \lr{ETSI} ارائه شده است، سعی دارند تا با آوردن بخشی از منابع محاسباتی به نزدیکی لبه شبکه، این مشکل را تا حدی برطرف کنند. علاوه بر تمایل دستگاه‌های لبه شبکه به کمتر شدن این فاصله و به عبارتی «کشش» منابع محاسباتی توسط آن‌ها به منظور افزایش کیفیت سرویس، شرکت‌های ارائه‌دهنده خدمات ابری نیز تمایل دارند تا با «فشردن» بخشی از منابع محاسباتی خود به لبه شبکه، بار محاسباتی و هزینه‌های تجهیزاتی خود را کاهش دهند. \cite{edgevisions}
\newpage
یک امر مهم در پیاده‌سازی کارآمد پردازش لبه‌ای، طراحی استراتژی‌های تخلیه وظایف هوشمند و موثر است. این استراتژی‌ها نحوه تخصیص منابع توسط دستگاه کاربر\LTRfootnote{User Equipment} را مشخص می‌کنند و این امکان را به دستگاه کاربر می‌دهند تا درباره تخلیه یا عدم تخلیه وظایف محاسباتی در طول زمان تصمیم بگیرد. \\

در این مقاله روشی برای بدست آوردن استراتژی تخلیه بهینه ارائه می‌نماییم که مبتنی بر زنجیره مارکوف گسسته-زمان و برنامه‌ریزی خطی می‌باشد. روش پیشنهادی گسترشی بر روش ارائه شده در \cite{Liu} می‌باشد. نوآوری و مزیت اصلی روش پیشنهادی نسبت به مقاله ذکر شده قابلیت پشتیبانی از وظایف با نیازمندی‌های پردازشی و شبکه‌ای متفاوت (وظایف ناهمگون) می‌باشد. یک انگیزه اصلی از این گسترش، تنوع وظایف محاسباتی در محیط‌های اینترنت اشیا بوده است. به طوری دقیق‌تر، در بسیاری از پژوهش‌های مرتبط با تخلیه وظایف در اینترنت اشیا، وظایف به دو دسته «سبک» و «سنگین» تقسیم شده اند. \cite{yousefpour} \cite{tran} برای نمونه در یک اتومبیل خودران، وظیفه پردازش اطلاعات تصاویر و سنسورها به منظور راندن خودرو یک وظیفه سنگین محسوب می‌شود. در حالی که وظیفه‌ای مانند روشن کردن سیستم گرمایشی خودرو بر حسب داده‌ی سنسور دما، یک وظیفه سبک محسوب می‌شود. \\

ادامه این مقاله به چهار فصل تقسیم شده است. در فصل ۲ مروری بر پژوهش‌های انجام شده در حوزه استراتژی‌های تخلیه وظایف در پردازش لبه‌ای صورت گرفته است. در فصل سوم روش پیشنهادی و نحوه پیاده‌سازی آن ارائه شده است. علاوه بر این، ساختار نرم‌افزاری\LTRfootnote{Framework} جدیدی مبتنی بر زبان \lr{Kotlin} ارائه شده است که این امکان را به کاربران و پژوهشگران می‌دهد تا استراتژی تخلیه بهینه را به ازای سیستم دلخواه خود محاسبه نمایند و آن استراتژی را با سایر استراتژی‌های پایه\LTRfootnote{Baseline} مقایسه نمایند. در فصل چهارم به طور جامع به آزمایش و شبیه‌سازی روش ارائه شده در سناریوهای مختلف پرداخته شده است. در انتها، در فصل ۵ یک جمع‌بندی کلی از تمامی مطالب ارائه شده است و پیشنهاداتی نیز برای گسترش روش پیشنهادی ارائه شده است.


\clearpage
